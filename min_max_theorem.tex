\documentclass[reqno]{amsart}
\usepackage[toc,page]{appendix}

\DeclareMathOperator*{\argmin}{arg\,min\,}
\DeclareMathOperator*{\argmax}{arg\,max\,}

\setlength{\parindent}{0pt}
\setlength{\parskip}{\baselineskip}

\numberwithin{equation}{section}

\title{The Rayleigh Quotient and Applications}
\author{Adam Williams}

\begin{document}

\maketitle

\section{The Rayleigh Quotient}

\subsection{Definition}

Let $A$ be an $n \times n$ real symmetric matrix.
Let $R_A : \mathbb R^n \setminus \{0\} \to \mathbb R$ be the \textbf{Rayleigh
quotient:}
$$
    R_A(x) = \frac{x^T Ax}{x^T x}.
$$
First we note that $R_A$ is invariant under scaling $(R_A(cx) = R_A(x)$, and so
it can be viewed as function on the unit sphere:
\begin{align*}
    R_A : S^{n-1} \to \mathbb{R} \\
    R_A(x) = x^T A x.
\end{align*}
Furthermore, $R_A(x)$ is a weighted average of eigenvalues of $A$, weighted by
the size of the component of $x$ along each eigenvalue: Letting $\lambda_1
\le \cdots \le \lambda_n$ be the eigenvalues of $A$, with corresponding
(orthonormal) eigenvectors $u_1, \ldots, u_n$, we have
\begin{equation}
    R_A(x) = \sum_{i=1}^n \lambda_i (u_i^T x)^2,
\label{eigen}
\end{equation}
where $u_i^Tx$ is the component of $x$ along $u_i$.

\subsection{Stationary Points and Global Extrema}

Suppose we wish to find the stationary points of $R_A$.
Using the method of Lagrange multipliers,
we wish to find the critical points of the Lagrange function
$$
    \mathcal L(x, \lambda) = R_A(x) - \lambda (x^T x - 1).
$$
Setting $\frac{ d \mathcal L(x)}{d x} = 0$ and straightforward calculation
yield $Ax = \lambda x$, meaning that the eigenvectors of $A$ are the
stationary points of $R_A$, and the corresponding eigenvalues are
the corresponding stationary values. In particular,
$$
    u_n = \argmax_x R_A(x)
$$
and
$$
    \lambda_n = R_A(u_n) = \max_x R_A(x).
$$
In other words, the Rayleigh quotient attains its maximum value at $u_n$,
with that value being $\lambda_n$. Similarly, $R_A$ attains its minimum
value of $\lambda_1$ at $u_1$.

Note that this can also easily be derived from (\ref{eigen}).

\subsection{Stationary Points}

We have shown that the maximum and minimum eigendirections maximize and
minimize the Rayleigh quotient, respectively. Now we consider any
eigendirection $(u_k, \lambda_k)$, and consider the subspace
orthogonal to the higher eigendirections: $U = \operatorname{span}(u_1,
\ldots, u_k)$. For any $x \in U$, we have $u_i^T x = 0$ for $i > k$, and
so
$$
    R_A(x) = \sum_{i=1}^k \lambda_k (u_i^T x).
$$
Clearly, this expression is maximized when $x = u_k$, with maximum

\section{Applications}

\subsection{Principal Component Analysis}

\subsection{Princial Curvatures}

\subsection{Averaging Rotations}

\subsection{Point Triangulation}

\end{document}
