\documentclass{amsart}
\usepackage[toc,page]{appendix}
\usepackage{amsmath}

\DeclareMathOperator*{\argmin}{arg\,min\,}
\DeclareMathOperator*{\argmax}{arg\,max\,}

\title{Gauss-Newton Optimization}

\begin{document}

\maketitle

\tableofcontents

\section{Introduction}

\section{General Unconstrained Optimization in Euclidean Spaces}
\subsection{The General Unconstrained Problem}
\subsection{The Newton Direction}
\subsection{Trust Region Methods}
\subsection{Convergence Theory}

\section{Gauss-Newton in Euclidean Spaces}
\subsection{The Gauss-Newton Method}
\subsection{Maximum Likelihood Formulation}
\subsection{Levenberg-Marquardt Damping}
\subsection{Robust Loss Functions}
\subsection{Convergence Theory}

\section{Extension to Manifolds}

\section{Implementation Concerns}
\subsection{Sparsity of Large-Scale Problems}
\subsection{Variable Elimination Orderings}
% ^ see wikipedia pages for Variable Elimination and Factor graph
\subsection{Gauge Freedom}
\subsection{Poorly Conditioned Problems}
\subsection{Observability}

\appendix

\section{Linear Least-Squares Methods}
\subsection{The Linear Problem}

\emph{Generalized linear least-squares} is a technique for solving the problem
\label{glls}
\begin{equation}
    x^* = \argmin_{x \in \mathbb R^n} \frac12 \| Ax - y\|_\Sigma^2,
\end{equation}
where $A \in \mathbb R^{m \times n}$, $y \in \mathbb R^m$, $\Sigma \in \mathbb R^{m \times m}$
is a symmetric, positive-definite matrix, and $\| z \|_\Sigma = (z^T \Sigma^{-1} z)^{1/2}$ is the
Mahalanobis norm on $\mathbb R^m$ induced by $\Sigma$. Differentiating (\ref{glls}) with respect
to $x$ and setting to zero yields the \textbf{normal equations}
\label{normal}
\begin{equation}
    A^T \Sigma^{-1} A x = A^T \Sigma^{-1} y.
\end{equation}
The matrix $A^T \Sigma^{-1} A$ is non-singular if and only if $A$ has full column rank. In this
case, the problem is said to be \emph{well-posed}, or \emph{observable}.
Clearly, if the problem is observable, then there is an analytic solution
$x^* = (A^T \Sigma^{-1} A)^{-1} A^T \Sigma^{-1} y$. We discuss this case first.

\subsection{Matrix Factorization Solutions}
Assuming that $A^T \Sigma^{-1} A$ is non-singular, solving (\ref{normal}) amounts to inverting a
symmetric positive-definite matrix. There are various methods for accomplishing this:
\begin{itemize}
    \item A real matrix $M$ is symmetric and positive-definite if and only if it admits
    a \textbf{Cholesky decomposition}
    $$
    M = LL^T,
    $$
    where $L$ is a non-singular, lower-triangular matrix. Using Cholesky decomposition to solve
    $Mx = b$ goes as follows:
    \begin{enumerate}
        \item Factor $M = LL^T$.
        \item Solve $Ly = b$ for $y$ by forward substitution.
        \item Solve $L^Tx = y$ for $x$ by back substitution.
    \end{enumerate}
    The Cholesky decomposition has a drawback when used for GLLS, namely that
    $$
    \kappa(A^T \Sigma^{-1} A) \sim \kappa(A)^2,
    $$
    where $\kappa(A) = \sigma_{\max}(A) / \sigma_{\min}(A) = \| A^{-1}\| \|A\|$ is the condition
    number of the matrix, measuring the maximum possible error in $y$ relative to $x$ given
    $y = Ax$. Hence
    a poorly conditioned $A$ can lead easily to a very poorly conditioned $A^T \Sigma^{-1} A$, which
    can be a problem when solving numerical problems.

    \item 
\end{itemize}

\subsection{Conjugate Gradient Solution}
\subsection{Regularization of Degenerate Problems}

\section{Lagrange Multipliers}

\section {$M$-Estimators}

\section{Matrix Calculus}

\section{Differential Geometry}
\subsection{Differentiable Manifolds}
\subsection{Riemannian Geometry}
\subsection{The Exponential Map and Normal Coordinates}

\section{Lie Groups}

\begin{thebibliography}{999}

\bibitem{nocedal}
Nocedal \& Wright,
\emph{Numerical Optimization}.
Springer Series In Optimization Research,
Second Edition,
2006.

\end{thebibliography}

\end{document}
